\documentclass[a4paper]{scrartcl}

\usepackage[utf8]{inputenc}
\usepackage[brazil]{babel}
\usepackage{siunitx}
\usepackage[siunitx, RPvoltages, betterproportions]{circuitikz}

\title{CircuiTikz}
\date{}

\begin{document}

\maketitle

\section{Instruments}

Instruments are immune to rotation. Take a look.

\begin{circuitikz}
    \foreach \a in {0,45,...,350}{
        \draw (0,0) to[oscope] (\a:3);
    }
\end{circuitikz}

Some examples with voltmeter and ammeter.

\begin{circuitikz}[american]
    \draw (0,0) -- ++(1,0) to[R] ++(2,0)
    to[smeter, t=A, i=$i$] ++(0,-2) node[ground]{};
    \draw (1,0) to[smeter, t=V, v=$v$] ++(0,-2) node[ground]{};
\end{circuitikz}

\begin{circuitikz}
    \draw (0,0) -- ++(1,0) to[R] ++(2,0)
    to[qiprobe, l=$i$] ++(0,-2) node[ground]{};
    \draw (1,0) to[qvprobe, l=$v$] ++(0,-2)
    node[ground]{};
\end{circuitikz}

\section{Beginning}

We begin our journey with this kind of circuit. It is the first kind of electrical circuit you study, usually, when you are a child/teenager and when you are in early days of university or college.

\begin{circuitikz}[american]
    \draw (0,0) to[isource, l=$I_0$] (0,3)
    to[short, -*, f=$I_0$] (2,3)
    to[R=$R_1$, f>_=$i_1$] (2,0)
    -- (0,0);
    \draw (2,3) -- (4,3)
    to[R=$R_2$, f>_=$i_2$] (4,0)
    to[short] (2,0);
    %mixing tikz elements
    \draw[red] (1.5, 2.5) rectangle (4.5, 3.5)
        node[pos=0.5, above]{KCL};
\end{circuitikz}

\begin{circuitikz}[european, voltage shift=0.5]
    \draw (0,0) to[isource, l=$I_0$, v=$V_0$] (0,3)
    to[short, -*, f=$I_0$] (2,3)
    to[R=$R_1$, f>^=$i_1$] (2,0)
    -- (0,0);
    \draw (2,3) -- (4,3)
    to[R=$R_2$, f<_=$i_2$] (4,0)
    to[short] (2,0);
\end{circuitikz}

\begin{circuitikz}[american]
    \draw (0,0) to[isource, l=$I_0$] (0,3)
    to[short, -*, i=$I_0$] (2,3)
    to[R=$R_1$, i>^=$i_1$] (2,0)
    -- (0,0);
    \draw (2,3) -- (4,3)
    to[R=$R_2$, i<_=$i_2$] (4,0)
    to[short] (2,0);
\end{circuitikz}

\begin{circuitikz}[american]
    \draw (0,0) to[isource, l=$I_0$] (0,3)
    to[short, -*, i=$I_0$] (2,3)
    to[R=$R_1$, i=$i_1$] (2,0)
    -- (0,0);
    \draw (2,3) -- (4,3)
    to[R=$R_2$, i=$i_2$] (4,0)
    to[short] (2,0);
\end{circuitikz}

\begin{circuitikz}[american]
    \draw (0,0) to[isource, l=$I_0$] (0,3) -- (2,3)
    to[R=$R_1$] (2,0) -- (0,0); 
    \draw (2,3) -- (4,3)
    to[R=$R_3$] (4,0) -- (2,0);
\end{circuitikz}

\begin{circuitikz}
    \draw (0,0) to[R=2, i=?, v=84] (2,0)
    -- (2,2) to[V<=84] (0,2)
    -- (0,0);
\end{circuitikz}

\begin{circuitikz}
    \draw (0,0) to[isource] (0,3) -- (2,3)
    to[R] (2,0) -- (0,0); 
\end{circuitikz}

\section{Path-style components}

Most path-style components can be used as a node-style component; to access them, you add a shape to the main name of component (for example, diodeshape).

\begin{circuitikz}
    \draw (0,0) to[potentiometer, name=P, mirror] ++(0,2);
    \draw (P.wiper) to[L] ++(2,0); 
\end{circuitikz}

\section{Node-style components}

Some op-amps to play with

\begin{circuitikz}[scale=0.7, transform shape]
    \draw (0,0) node[op amp] {OA1};
    \draw (0,3) node[op amp] {OA2};
    %xscale e yscale also affect text labels
    %so they need to be un-scaled \scalebox{-1}{1}
    %another option is to use one of these 3 helper macros
    %\ctikzflipx{}
    %\ctikzflipy{}
    %\ctikzflipxy{}
    \draw (3,3) node[op amp, xscale=-1] {OA3};
    \draw (3,0) node[op amp, xscale=-1] {OA4};
\end{circuitikz}

\begin{circuitikz}[scale=0.7, transform shape]
    \draw (0,0) node[op amp] {OA1};
    \draw (0,3) node[op amp] {OA2};
    \draw (3,3) node[op amp, xscale=-1] {\ctikzflipx{OA3}};
    \draw (3,0) node[op amp, xscale=-1] {\ctikzflipx{OA4}};
\end{circuitikz}

\section{Resistive bipoles}

Resistors everywhere.

\begin{circuitikz}
    \ctikzset{
        resistors/width=1.5,
        resistors/zigs=9,
    }
\end{circuitikz}

\section{Inductors}

They can be really cute.

\begin{circuitikz}
    [
        longL/.style = {
            cute choke,
            inductors/scale = 0.75,
            inductors/width = 1.6,
            inductors/coils = 9,
        },
    ]
    \draw (0, 1.5) to[L, l=$L$] ++(4,0); 
    \draw (0, 0) to[longL, l=$L$] ++(4,0); 
    \ctikzset{inductors/scale=1.5, inductor=american}
    \draw (0, -1.5) to[L, l=$L$] ++(4,0);
\end{circuitikz}

Customizing the chokes (which come only in the cute style)

\begin{circuitikz}
    \draw (0,0) to[cute choke] ++(3,0); 
    \draw (0,-1) to[cute choke, twolineschoke] ++(3,0); 

    \ctikzset{bipoles/cutechoke/cthick=2, twolineschoke}

    \draw (0,-2) to[cute choke ] ++(3,0); 
    \draw (0,-3) to[cute choke, onelinechoke] ++(3,0); 
\end{circuitikz}

\section{Transformers}

We work a lot with transformers. Let's learn how to draw them properly.

\begin{circuitikz}
   \draw
   (0,0) node[transformer] (T) {}
   (T.A1) node[anchor=east] {A1}
   (T.A2) node[anchor=east] {A2}
   (T.B1) node[anchor=west] {B1}
   (T.B2) node[anchor=west] {B2}
   (T.base) node{K}
   (T.inner dot A1) node[circ]{}
   (T.inner dot B2) node[circ]{}
   ;
\end{circuitikz}

\begin{circuitikz}
    \ctikzset{
        quadpoles/transformer/width=1,
        quadpoles/transformer/height=2,
        %quadpoles style=inward (default),
    }
    \draw
    (0,0) node[transformer] (T) {}
    (T.base) node{K}
    (T.inner dot A1) node[circ]{}
    (T.inner dot B2) node[circ]{}
    ;
\end{circuitikz}
    
\begin{circuitikz}
    \ctikzset{
        quadpoles/transformer/width=1,
        quadpoles/transformer/height=2,
        inductor=cute,
        quadpoles style=inline,
    }
    \draw
    (0,0) node[transformer] (T) {}
    (T.base) node{K}
    (T.inner dot A1) node[circ]{}
    (T.inner dot B2) node[circ]{}
    ;
\end{circuitikz}

\begin{circuitikz}[american]
    \begin{scope}
        \ctikzset{
        transformer L1/.style={
            inductors/coils=1,
            inductors/width=0.2,
        }
    }
        \draw (0,0) node[transformer](T1){};
    \end{scope}
    \draw (3,0) node[transformer](T2){};
\end{circuitikz}

\begin{circuitikz}
    \ctikzset{
        quadpoles style=inline,
        cute inductors,
        transformer L1/.style={
            inductors/coils=2,
            inductors/width=0.2,
        }
    }
    \draw (0,0) node[transformer](T1){};
    \ctikzset{
        quadpoles style=inward,
        cute inductors,
        transformer L2/.style = {
            inductors/coils=7,
            inductors/width=1.0,
        }
    }
    \draw (3,0) node[transformer](T2){};
\end{circuitikz}

%\begin{circuitikz}
%    %oosourcetrans and oosource have been added by user @olfline
%    %May 3,2020. version?
%    %My version of circuitikz is deprecated
%    \draw (0,0) to[oosourcetrans, prim=zig, sec=delta, o-] ++(2,0)
%        to[oosourcetrans, prim=delta, sec=wye, -o] ++(0,-2)
%        to[ooosource, prim=wye, sec=zig, tert=delta] (0,0);
%\end{circuitikz}

\begin{circuitikz}
    \ctikzset{
        quadpoles style=inward,
        cute inductors,
        transformer L1/.style={
            inductors/coils=9,
            inductors/width=1.0,
        },
        transformer L2/.style={
            inductors/coils=3,
            inductors/width=0.4,
        },
    }
    \draw (0,0) node[transformer] (T1) {};
\end{circuitikz}


\end{document}
